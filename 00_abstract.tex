\section{Abstract}\label{abstract}

\subsection*{Background}\label{background}

The healthcare industry is seeing a digital revolution, resulting in an
ever-growing influx of data. This transformation creates an urgent need
for efficient and automated data visualization tools. Visual Viper (VV)
aims to meet this demand by offering an automated Python library that
streamlines the complex and often time-consuming process of creating
data visualizations for scientific communication.

\subsection*{Aim}\label{aim}

The aim of this study is to outline the development of VV, assess its
performance and adaptability, explore its modular design and development
methodologies, and establish its practical applications in healthcare
research.

\subsection*{Methods}\label{methods}

Built using Python, VV employs Vega-Lite for high-level interactive
graphics. The library is structured with modular, extensible
architecture, developed with object-oriented programming (OOP) and
test-driven development (TDD) practices. Docker containerization ensures
a consistent development environment, and GitLab version control,
aligned with Semantic Versioning, streamlines collaborative development.
Native CI/CD capabilities of GitLab further enrich the development
process. VV operates environment-agnostically and offers serverless
deployment options.

\subsection*{Results}\label{results}

VV includes various interconnected components, each responsible for
specific tasks ranging from data retrieval to chart rendering. Four main
classes (\textquotesingle DatasetBuilder\textquotesingle,
\textquotesingle ChartNotationBuilder\textquotesingle, `ChartRenderer',
and \textquotesingle ChartDeployer\textquotesingle) encapsulate the
respective functionalities, thus aiding in code maintenance and
extension.

Evaluation metrics, captured using Monday.com and Python's time library,
showed that while VV required a longer initial setup time (2h vs. 0.5h),
it outperformed manual methods in "Time-to-Final-Chart" (2h9min vs.
14h54min) for a project involving 72 spreadsheets. Adjusted metrics
accounting for task fatigue and human intervention also favor VV,
especially for larger and ongoing projects.

VV effectively minimizes manual labor, ensures data visualization
consistency and fosters best practices in scientific communication.
Current limitations include a focus on mostly specific organizational
workflows and visualizations.

\subsection*{Conclusion}\label{conclusion}

VV presents a robust and customizable solution for automating data
visualization. It holds promise for significantly enhancing scientific
communication efficiency within the healthcare sector, with its modular
and scalable design paving the way for future developments.

\newpage

\section{Resumo}\label{resumo}

\subsection*{Contexto}\label{contexto}

O sector da saúde está a ser alvo de uma revolução digital que resulta
no aumento crescente de dados. Esta transformação cria uma necessidade
urgente de ferramentas de visualização de dados eficientes e
automatizadas. Visual Viper (VV) tem em vista responder a esta
necessidade, oferecendo uma biblioteca Python que automatiza e
simplifica o processo complexo e muitas vezes demorado de criação de
visualizações de dados para comunicação científica.

\subsection*{Objetivo}\label{objetivo}

O objetivo deste trabalho é descrever o desenvolvimento do VV, avaliar o
seu desempenho e adaptabilidade, explorar o seu desenho modular e
metodologias de desenvolvimento utilizadas, bem como estabelecer as suas
aplicações práticas na investigação em saúde.

\subsection*{Métodos}\label{muxe9todos}

Construído com recurso à linguagem Python, o VV emprega Vega-Lite para
gráficos interativos de alto nível. A biblioteca é estruturada com
arquitetura modular, extensível, desenvolvida com práticas de
programação orientada a objetos (OOP) e desenvolvimento orientado por
testes (TDD). A utilização de contentores Docker garante um ambiente de
desenvolvimento consistente, e o controle de versão utilizando GitLab em
conjugação com o sistema de Versionamento Semântico, simplifica o
desenvolvimento colaborativo. As capacidades nativas de CI/CD do GitLab
enriquecem ainda mais o processo de desenvolvimento. O VV opera de forma
agnóstica de ambiente e permite opções de implementação sem servidor.

\subsection*{Resultados}\label{resultados}

O VV inclui vários componentes interconectados, cada um responsável por
tarefas específicas que vão desde a leitura de dados até à renderização
de gráficos. Quatro classes principais -
\textquotesingle DatasetBuilder\textquotesingle,
\textquotesingle ChartNotationBuilder\textquotesingle, `ChartRenderer',
e \textquotesingle ChartDeployer\textquotesingle{} - encapsulam as
respectivas funcionalidades, facilitando a manutenção e extensão do
código.

As métricas de avaliação, capturadas com recurso ao Monday.com e a
biblioteca `time' do Python, mostraram que embora o VV tenha exigido um
tempo de configuração inicial mais longo (2h vs. 0.5h), superou os
métodos manuais em "Time-to-Final-Chart" (2h9min vs. 14h54min) para um
projeto que envolvia 72 folhas de cálculo. Métricas ajustadas que
incluem o efeito da fadiga da tarefa e a intervenção humana, também
favorecem o VV, especialmente para projetos maiores e contínuos.

O VV efetivamente minimiza o trabalho manual, garante a consistência da
visualização dos dados, e promove boas práticas de comunicação
científica. Atualmente, as limitações incluem um foco em fluxos de
trabalho e visualizações tendencialmente específicas da organização.

\subsection*{Conclusão}\label{conclusuxe3o}

O VV apresenta uma solução robusta e personalizável para a automação da
visualização de dados. Promete melhorar significativamente a eficiência
da comunicação científica dentro do setor de saúde, com seu design
modular e escalável abrindo caminho para desenvolvimentos futuros.
