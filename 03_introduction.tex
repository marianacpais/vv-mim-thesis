\section{Introduction}\label{introduction}

Healthcare is generating an unprecedented amount of data due to the rise
in digital technology
\cite{1,2}. This data,
ranging from patient records to complex genetic information, holds value
for various studies, including those related to real-world evidence.
However, the sheer volume of this data makes manual chart generation and
updating increasingly impractical. As such, automation is becoming
essential for efficient data interpretation in healthcare.

As healthcare increasingly digitizes, the sector is inundated with a
complex array of data that professionals and researchers must make sense
of. While visualization tools exist, they often don\textquotesingle t
address the specific needs of healthcare data or scale well with big
data challenges
\cite{3,4}. Moreover,
the manual effort involved in using these tools remains significant.
Therefore, there\textquotesingle s a growing demand for an automated and
scalable solution capable of simplifying the generation and deployment
of relevant visualizations.

\subsection{Objectives}\label{objectives}

The aim of this project is to conceive, architect, develop and evaluate
Visual Viper (VV), a Python library aimed to automate the creation of
data visualizations in the healthcare sector. This work will provide a
description of each phase, from initial requirement gathering and system
architecture design to coding, testing, and evaluation. Limitations will
be discussed, along with suggestions for future enhancements.

To provide a comprehensive understanding of the scope of this project,
the following objectives are enumerated:

\begin{itemize}
\item
  Conduct an initial requirement analysis to identify the specific needs
  and constraints that VV aims to address.
\item
  Outline the architecture of VV while adhering to best practices in
  software development.
\item
  Implement the designed architecture of VV, emphasizing its modular and
  extensible nature.
\item
  Apply and critically analyze software development methodologies such
  as object-oriented programming and test-driven development in the
  creation of VV, considering their impact on the code\textquotesingle s
  quality, maintainability, and extensibility.
\item
  Implement, test, and evaluate the features that VV offers for data
  retrieval, transformation, and visualization, with a specific focus on
  retrieving data from Google Sheets, creating Forest Plots, and
  deploying visualizations to Miro Board and Google Drive.
\item
  Conduct performance testing on VV to assess its efficiency and
  scalability, especially when handling large healthcare datasets.
\item
  Review and identify areas of improvement within the current version of
  VV, setting the stage for future iterations and enhancements.
\item
  Assess the tool\textquotesingle s success in automating the data
  visualization process in healthcare research, measuring its
  effectiveness in facilitating scientific communication.
\end{itemize}

The project seeks to fill a critical gap in the existing tools for
automating the generation of healthcare data visualization. By
automating the often labor-intensive and complex process of generating
custom visualizations, VV aims to significantly improve the efficiency
of scientific communication in healthcare. It also introduces a modular
and extensible architecture, enabling the library to adapt to diverse
data sources and evolving visualization needs, thereby extending its
lifespan and relevance.

\subsection{Thesis Overview}\label{thesis-overview}

This thesis is organized as follows:

\begin{itemize}
\item
  \textbf{Introduction}: This section encompasses the background of the
  study, problem statement, purpose, research objectives, and
  justification. It serves to establish the context and significance of
  the research.
\item
  \textbf{Background}: This section provides a concise assessment of the
  literature relevant to data visualization and software development paradigms.
\item
  \textbf{Methodology}: This section details the methodologies adopted.
  It covers aspects like requirement analysis, user stories, and scope
  of the project. Core principles such as modularity, extensibility, and
  usability are also elaborated in separate subsections.
\item
  \textbf{Development Approach}: Offers an overview of the development
  approach and programming paradigms employed. It discusses possible
  development approaches and discusses the use of Object-Oriented
  Programming and Test-Driven Development.
\item
  \textbf{Development Environment and Tools}: In this section, the
  various tools utilized during the development process are covered. This includes aspects of Docker containerization, version
  control through GitLab, and CI/CD pipelines. Furthermore, this section
  elucidates the build automation process and discusses the Makefile in
  detail.
\item
  \textbf{System Architecture}: Provides an in-depth description of the
  system's architecture. It discusses the high-level architecture, key
  classes, component interactions, and data flows among components.
\item
  \textbf{Implementation Details}: This section delves into the
  technical nuances of the project\textquotesingle s implementation.
\item
  \textbf{Workflow Demonstration}: This section provides a demonstration
  of how the VV system operates in a real-world context. The aim is to
  convey both the utility and the user experience of the system.
\item
  \textbf{Evaluation Results}: This section analyzes
  VV\textquotesingle s performance on the specific use cases of Google
  Sheets data retrieval, Forest Plots creation, and deployment to Miro
  Board and Google Drive.
\item
  \textbf{Discussion}: This section serves as a platform to review the
  research findings and to propose future recommendations.
\item
  \textbf{Conclusion}: Summarizes the research and outlines the
  contributions made by the study.
\item
  \textbf{References}: Lists all sources cited throughout the document.
\end{itemize}
