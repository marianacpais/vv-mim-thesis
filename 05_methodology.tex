\section{Methodology}\label{methodology}

The methodology section serves as a roadmap detailing the design,
development, and evaluation of the VV Python library. The objective here
is to offer comprehensive insights into the technical aspects of VV,
elucidating the rationale behind various design and architectural
choices, as well as the methods used for implementation and assessment.
Given that this library aims to bridge a gap in healthcare data
visualization, especially in handling big data and providing
customizable solutions for automation, it is crucial to understand the
techniques and technologies that make it both functional and scalable.

This section will start by explaining the basic ideas behind the VV
project. Then, we\textquotesingle ll get into the actual development
aspects, including our use of Object-Oriented Programming (OOP) and
Test-Driven Development (TDD). Lastly, we will explore how the library
was evaluated, describing the metrics and methods used during the
evaluation phase.

\subsection{Requirement Analysis}\label{requirement-analysis}

This section outlines the key functions and quality features expected of
the VV system. It provides a set of clear requirements that will guide
the design and implementation stages of the project. To enhance the
system\textquotesingle s effectiveness and ease of use, we present
selected use cases that illustrate how VV will interact with other
systems for better integration in the broader data visualization
landscape. In short, this section sets the foundational requirements
that will guide the development efforts.

\subsubsection{User Stories}\label{user-stories}

User stories serve as a vehicle for capturing product functionality from
the end user\textquotesingle s perspective. These stories encapsulate
discrete system features in a format that is easy to read and understand
by both non-technical stakeholders and the development team
\cite{37}\cite{38}. In the
context of VV, a system designed to automate the rendering of graphical
charts from clinical research data, the user stories described here are
aimed to outline the essential features and functionalities that satisfy
the needs of different roles involved in clinical research.

\paragraph{Scope}\label{scope}

The following user stories are specifically tailored to the needs of
clinical researchers, medical writers, data analysts, and system
administrators who are the key stakeholders of the VV system. They focus
on tasks related to data visualization, report generation, and system
management within the context of clinical research.

\paragraph{Stakeholder Definitions}\label{stakeholder-definitions}

\begin{itemize}
\item
  \textbf{Clinical Researcher}: A professional conducting clinical
  studies.
\item
  \textbf{Medical Writer}: A professional responsible for creating
  documents that describe research results, product use, and other
  scientific dissemination outlets.
\item
  \textbf{Data Analyst}: A person responsible for interpreting complex
  clinical data sets.
\item
  \textbf{Data Scientist}: A professional who uses scientific methods,
  processes, algorithms, and systems to extract insights and knowledge
  from structured and unstructured data.
\item
  \textbf{System Administrator}: A person responsible for managing and
  maintaining the system infrastructure, including VV.
\end{itemize}

\paragraph{Story 1: Batch Rendering of Clinical
Charts}\label{story-1-batch-rendering-of-clinical-charts}

\begin{itemize}
\item
  \textbf{As a} clinical researcher/data scientist,
\item
  \textbf{I want} VV to batch render multiple charts from automatically
  generated clinical reports,
\item
  \textbf{So that} a large volume of data can be visually represented
  quickly and efficiently.
\item
  \textbf{Acceptance Criteria}:

  \begin{itemize}
  \item
    System should be able to accept multiple clinical reports as input.
  \item
    System should be able to render charts in batches without manual
    intervention.
  \item
    Rendered charts should accurately represent the data from the
    clinical reports.
  \item
    System should provide an option for selecting the types of charts to
    be rendered (bar, line, etc.).
  \item
    Batch rendering process should complete within a reasonable time
    frame (e.g., under 5 minutes for 50 reports).
  \end{itemize}
\end{itemize}

\paragraph{Story 2: Deployment to Miro for
Triage}\label{story-2-deployment-to-miro-for-triage}

\begin{itemize}
\item
  \textbf{As a} clinical researcher/data scientist,
\item
  \textbf{I want} VV to deploy rendered charts directly to specific
  boards in Miro,
\item
  \textbf{So that} they can be quickly triaged alongside tabular
  reports.
\item
  \textbf{Acceptance Criteria}:

  \begin{itemize}
  \item
    System should integrate with Miro API.
  \item
    System should be able to send rendered charts to specified Miro
    boards.
  \item
    Rendered charts should appear on the Miro boards in a layout that
    facilitates triage.
  \item
    Charts should be deployed to Miro boards without manual
    intervention.
  \end{itemize}
\end{itemize}

\paragraph{Story 3: Export Charts for Research
Documents}\label{story-3-export-charts-for-research-documents}

\begin{itemize}
\item
  \textbf{As a} medical writer,
\item
  \textbf{I want} VV to export rendered charts in formats suitable for
  academic manuscripts, posters, and other research documents,
\item
  \textbf{So that} the visual data complements the written content.
\item
  \textbf{Acceptance Criteria}:

  \begin{itemize}
  \item
    System should offer multiple export formats such as PNG, JPEG, SVG,
    etc.
  \item
    Exported charts should maintain high resolution and quality.
  \item
    System should allow for batch export of multiple charts.
  \end{itemize}
\end{itemize}

\paragraph{Story 4: Inclusion of Supplementary
Material}\label{story-4-inclusion-of-supplementary-material}

\begin{itemize}
\item
  \textbf{As a} clinical researcher/data scientist,
\item
  \textbf{I want} VV to render charts that can be included as
  supplementary material when publishing,
\item
  \textbf{So that} we can increase the transparency of our research.
\item
  \textbf{Acceptance Criteria}:

  \begin{itemize}
  \item
    System should allow rendering of charts that are suitable for
    supplementary material in terms of quality and resolution.
  \item
    System should allow for easy categorization or labeling of such
    charts for supplementary material.
  \item
    Charts should be exportable in a format accepted by major research
    publications.
  \end{itemize}
\end{itemize}

\paragraph{Story 5: Automated Data
Retrieval}\label{story-5-automated-data-retrieval}

\begin{itemize}
\item
  \textbf{As a} data analyst,
\item
  \textbf{I want} VV to automatically retrieve data from predefined
  clinical report formats,
\item
  \textbf{So that} I don\textquotesingle t have to manually input data
  for chart rendering.
\item
  \textbf{Acceptance Criteria}:

  \begin{itemize}
  \item
    System should be able to identify and read predefined clinical
    report formats.
  \item
    System should accurately extract relevant data fields from these
    reports.
  \item
    Data retrieval should happen automatically through API calls.
  \end{itemize}
\end{itemize}

\paragraph{Story 6: Customization of Chart
Types}\label{story-6-customization-of-chart-types}

\begin{itemize}
\item
  \textbf{As a} clinical researcher/data scientist,
\item
  \textbf{I want} to specify the type of chart (bar, line, scatter,
  etc.) VV should render,
\item
  \textbf{So that} the chart is most appropriate for the data being
  represented.
\item
  \textbf{Acceptance Criteria}:

  \begin{itemize}
  \item
    System should offer a range of chart types (bar, forest plot,
    survival, etc.).
  \item
    Users should be able to easily select the desired chart through
    configuration.
  \item
    Rendered charts should accurately represent the selected chart type.
  \end{itemize}
\end{itemize}

\paragraph{Story 7: Logging and
Monitoring}\label{story-7-logging-and-monitoring}

\begin{itemize}
\item
  \textbf{As a} system administrator,
\item
  \textbf{I want} VV to keep logs of all chart rendering activities,
\item
  \textbf{So that} I can monitor system performance and troubleshoot
  issues.
\item
  \textbf{Acceptance Criteria}:

  \begin{itemize}
  \item
    System should maintain logs for each chart rendering activity.
  \item
    Logs should include timestamps, types of charts rendered, and any
    errors or warnings.
  \item
    Logs should be easily accessible for review and analysis.
  \end{itemize}
\end{itemize}

\paragraph{Story 8: Re-run Chart Rendering with Updated
Data}\label{story-8-re-run-chart-rendering-with-updated-data}

\begin{itemize}
\item
  \textbf{As a} clinical researcher/data scientist/medical writer,
\item
  \textbf{I want} to re-run chart rendering when new data is available,
\item
  \textbf{So that} my visual representations are always up-to-date.
\item
  \textbf{Acceptance Criteria}:

  \begin{itemize}
  \item
    System should allow for easy updating of data sources.
  \item
    Users should be able to initiate re-rendering without having to redo
    the entire setup.
  \end{itemize}
\end{itemize}

\subsubsection{Non-functional
Requirements}\label{non-functional-requirements}

The non-functional requirements for VV aim to outline the quality
attributes the system should possess. These are essential aspects that
define how well the system performs its functions rather than what
functions it performs. They encompass characteristics like modularity,
error handling, and auditability, among others. These requirements are
especially critical in ensuring that VV is not only functional but also
efficient, maintainable, and adaptable to various environments and
use-cases. Below is a list of the non-functional requirements we deem
essential for the system:

\paragraph{System Architecture}\label{system-architecture}

\begin{itemize}
\item
  \textbf{Modularity}: The system should be modular to allow for easier
  debugging and updating of individual components.
\item
  \textbf{Extensibility}: Designed in a way to easily allow the addition
  of new functionalities.
\end{itemize}

\paragraph{Usability and User
Experience}\label{usability-and-user-experience}

\begin{itemize}
\item
  \textbf{Configurability}: Users should be able to easily configure
  chart rendering options regardless of the environment (API, module,
  terminal).
\item
  \textbf{Environment Agnosticism}: Should be usable as an importable
  Python module, accessible via web API, or through the terminal.
\end{itemize}

\paragraph{Reliability}\label{reliability}

\begin{itemize}
\item
  \textbf{Error Handling}: The system should be able to gracefully
  handle errors and exceptions, providing useful error messages.
\end{itemize}

\paragraph{Maintenance and Support}\label{maintenance-and-support}

\begin{itemize}
\item
  \textbf{Documentation}: All code should be well-documented, and system
  documentation should be easily accessible for maintenance activities.
\item
  \textbf{Auditability}: Should provide logging features to keep track
  of data processing and rendering activities.
\end{itemize}

\subsection{Applied Technical Foundations and Development Paradigms}\label{technical-foundations-and-development-paradigms}

The objective of VV is the automation of data visualization, helping
with the challenges in handling large and complex data sets common in
healthcare. Concurrently, the project serves an educational purpose,
offering the developer a framework to explore and learn fundamental
software development paradigms. This educational aspect makes it crucial
to ensure that the project adheres to established coding practices and
methodologies, making it both a practical tool for data visualization
and a case study in applying robust software development principles.

The following sections will delve into the specifics of these
foundational principles, revealing how they guided the choices in
architecture and functionalities in VV.

\subsubsection{Modularity}\label{modularity}

In VV, modularity is a fundamental element guiding our design approach.
This ensures that each module is a self-contained unit with well-defined
interfaces, enhancing both reusability and portability, attributes
highly valued in specialized fields like healthcare informatics
\cite{39}.

\subsubsection{Object-Oriented Programming
(OOP)}\label{object-oriented-programming-oop}

In the VV library, OOP serves as a pivotal architectural choice, both
for the developer\textquotesingle s educational enrichment and the
system\textquotesingle s overall functionality and extensibility.
Employing OOP facilitates encapsulation, which allows for the bundling
of data and methods that operate on that data within single units or
classes.

OOP also uses inheritance, enabling code reusability and abstraction.
For instance, different types of charts, be it a bar chart, a forest
plot, or a survival plot, can be represented as individual classes.
These classes can contain methods to set chart properties, draw axes,
and render the data. Since each chart type may have common
characteristics such as a title or axes labels, inheritance allows these
shared features to be abstracted into a parent class. Specific chart
types can then inherit from this parent class, enabling them to reuse
common code while still allowing for their own specialized features.
Furthermore, different deployment targets, like cloud storage or Miro
boards, can also be abstracted into separate classes, encapsulating the
methods required for deploying visualizations to these locations. This
makes the system adaptable and easier to integrate with new deployment
options as needs evolve.

\subsubsection{Test-Driven Development
(TDD)}\label{test-driven-development-tdd}

TDD serves as a rigorous verification mechanism that aligns with the
project's objective of delivering a reliable and high-quality tool.
Based on the review on the impact of TDD on program design
and software quality, as well as the educational benefits for the
author, we have selected TDD as a methodology for our software
development project. This hands-on exposure is expected to be invaluable in future projects
and particularly beneficial when collaborating within larger teams that
also utilize TDD.

To implement TDD in this project, we selected pytest as the testing
library for its feature-rich environment, ease of use, and compatibility
with various Python frameworks. It provides detailed failure reports to
streamline debugging, and its straightforward syntax is especially
beneficial for those new to TDD
\cite{40}.

\subsection{Evaluation Metrics and
Methods}\label{evaluation-metrics-and-methods}

The evaluation phase for the VV Python library was designed to assess
both the functional capabilities of the library and its impact on
workflow efficiency. The key performance indicators (KPIs) used for this
evaluation were "Time to First Chart Draft" and "Time to Final Chart,"
designed to capture the time-efficiency gains enabled by the VV library.

\subsubsection{Time to First Chart
Draft}\label{time-to-first-chart-draft}

This metric captures the time needed from receiving the initial dataset
to generating the first draft of a chart. For the manual method, this
involves gathering values for relevant measures, preparing a Vega-Lite
JSON definition, populating the JSON with the data and adjusting
necessary parameters.

\subsubsection{Time to Final Chart}\label{time-to-final-chart}

This metric gauges the time from the receipt of the initial data to the
point where the chart is exported in the appropriate format (e.g., SVG)
and uploaded to a platform like Google Drive and included in a Miro
board for further analysis and comparison. This encompasses the entire
lifecycle of chart production and is intended to capture any efficiency
gains that may be achieved through the VV library.

\subsubsection{Data Sources for
Evaluation}\label{data-sources-for-evaluation}

The primary data source for these evaluations is time-tracking data from
MTG Research and Development Lab activities. This data focuses on chart
development for academic papers and is an integral part of our
methodology. It has been recorded using a tracker within the Monday.com
platform, which is the project management tool employed by the company
for all R\&D activities. This time-tracking data from past projects,
where chart generation was performed manually, serves as a comparative
baseline for evaluating the VV Python library\textquotesingle s
effectiveness.

\subsubsection{Simulation for Adjustment for Fatigue and Human
Intervention}\label{simulation-for-adjustment-for-fatigue-and-human-intervention}

To provide a comprehensive evaluation of the VV Python
library\textquotesingle s efficiency in chart creation, we extended our
analysis by including a simulation that includes considerations for task
fatigue and additional human intervention for validation. For this
exercise, we focused on the "Time-to-Final-Chart" metric, which captures
the total time needed to finalize a chart, accounting for all
adjustments and confirmations.

The analysis was conducted using R (version 4.2.3)
\cite{41}, and visualizations
were generated using the ggplot2 package
\cite{13}.
