\chapter{Discussion}\label{discussion}

\minitoc

The earlier chapters provided an in-depth look at the system
I\textquotesingle ve developed, focusing on its architecture, features,
and the evaluation metrics that attest to its performance. This
discussion aims to offer a comprehensive reflection on this work,
examining its current limitations, potential for future development, and
the broader implications it could have in academic and healthcare
contexts.

\section{Integration in Academic and Healthcare
Contexts}\label{integration-in-academic-and-healthcare-contexts}

The ability to dynamically create and update
charts like Forest Plots could be invaluable in both educational
settings and medical research. For example, the tool could be integrated
into academic courses focusing on statistical methods, epidemiology, or
healthcare management, offering students hands-on experience with data
visualization. In healthcare settings, the system could aid in real-time
data tracking and analytics, which is crucial in making timely and
data-backed decisions. The application\textquotesingle s modularity and
the possibility of developing specific plugins make it highly adaptable
to different academic and clinical use-cases.

\section{Deployment Options}\label{deployment-options}

As it currently stands, the system operates solely in a local
environment. While this setup serves its purpose for small-scale,
individual projects, it\textquotesingle s limited in terms of
scalability and ease of integration into larger workflows. Transitioning
to a cloud-based service could effectively address these limitations.

AWS Lambda offers an appealing solution for several reasons. First, it
eliminates the need to manage servers or clusters, allowing the focus to
remain on code execution. This is particularly beneficial because you
only pay for the computation time used, making it a cost-effective
choice. Lambda can also automatically respond to code execution requests
on any scale, from a few events per day to hundreds of thousands per
second, which makes it well-suited for projects with variable demand
\cite{50}.

\section{Limitations}\label{limitations}

One limitation of the current system is that the developed plugins are
inherently designed to suit the specific workflow requirements of the
company where the author works. This could pose challenges in adapting
the tool for more generalized use-cases. To enhance the
system\textquotesingle s utility across various applications, it would
be necessary to either develop additional plugins or modify the existing
ones to accommodate different configuration parameters.

Another significant limitation remains in terms of deploying charts that
handle vector graphics, which would allow researchers to fine-tune the
charts intuitively. We initially considered Figma as a potential
platform for deployment, but the Figma API is predominantly read-only.
It permits only writing comments but restricts manipulating graphical
elements directly. This gap opens up a possibility for future work in
finding or creating a more versatile platform for chart deployment.

\section{Planned Future
Developments}\label{planned-future-developments}

While our focus has been on the Forest Plot plugin due to its prominence
in our current large-scale projects, such as one that involves creating
360 Forest Plots, we acknowledge the need for additional chart types.
Upcoming releases could include plugins for survival charts, bar charts,
and Sankey diagrams.

To make the system more user-friendly, we aim to develop a Command Line
Interface (CLI). A CLI would streamline the user experience by providing
a straightforward way to configure various system parameters, ideally
reducing the initial setup time.

\section{Software Development Learning
Insights}\label{software-development-learning-insights}

Another important outcome of this project is the experience gained in
software development methodologies and best practices. While
architecting the system, there was an emphasis on employing effective
development paradigms and applying established design patterns. Overall,
the development process served as a practical case study in applying a
blend of software engineering principles, development paradigms, and
data structures to create a robust and scalable data visualization tool.
